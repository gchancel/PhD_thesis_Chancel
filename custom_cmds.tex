% !TeX root = ./0_Manuscript.tex

\tikzset{
    fc-term/.style={ draw, align=center, rounded rectangle },
    fc-proc/.style={ draw, align=center, rectangle },
    fc-dec/.style={ draw, align=center, diamond, aspect=1.65 },
    %fc-subproc/.style={ draw, rectangle },
    fc-loop-start/.style={ draw, align=center, signal, signal to=south },
    fc-loop-end/.style={ draw, align=center, signal, signal to=north },
    fc-yes/.style={-latex},
    fc-no/.style={{Circle[open, length=5pt]}-latex}
}

\newcommand{\tochap}[1]{\hyperref[#1]{Chapter~\ref{#1}}}
\newcommand{\tosec}[1]{\hyperref[#1]{Section~\ref{#1}}}
\newcommand{\totab}[1]{\hyperref[#1]{Table~\ref{#1}}}
\newcommand{\tofig}[1]{\hyperref[#1]{Figure~\ref{#1}}}
\newcommand{\tofigs}[0]{Figures~}

\newcommand{\todo}[1]{{\color{red}\textbf{[TODO: #1]}}}
\newcommand*\circled[1]{\tikz[baseline=(char.base)]{
            \node[shape=circle,draw,inner sep=1pt] (char) {#1};}}
\newcommand*\fillcircled[1]{\tikz[baseline=(char.base)]{
            \node[shape=circle, fill=black, text=white, draw=black, inner sep=0.1em] (char) {#1};}}

\newcommand\Tstrut{\rule{0pt}{2.6ex}}       % "top" strut
\newcommand\Bstrut{\rule[-0.9ex]{0pt}{0pt}} % "bottom" strut
\newcommand{\TBstrut}{\Tstrut\Bstrut}       % top&bottom struts

\DeclareUnicodeCharacter{24EA}{\circled{0}}
\DeclareUnicodeCharacter{2460}{\circled{1}}
\DeclareUnicodeCharacter{2461}{\circled{2}}
\DeclareUnicodeCharacter{2462}{\circled{3}}
\DeclareUnicodeCharacter{2463}{\circled{4}}
\DeclareUnicodeCharacter{2464}{\circled{5}}
\DeclareUnicodeCharacter{2465}{\circled{6}}
\DeclareUnicodeCharacter{2466}{\circled{7}}
\DeclareUnicodeCharacter{2467}{\circled{8}}
\DeclareUnicodeCharacter{2468}{\circled{9}}
\DeclareUnicodeCharacter{24B6}{\circled{A}}
\DeclareUnicodeCharacter{24B7}{\circled{B}}
\DeclareUnicodeCharacter{24B8}{\circled{C}}
\DeclareUnicodeCharacter{24B9}{\circled{D}}
\DeclareUnicodeCharacter{24BA}{\circled{E}}
\DeclareUnicodeCharacter{24BB}{\circled{F}}

\DeclareUnicodeCharacter{24FF}{\fillcircled{0}}
\DeclareUnicodeCharacter{2776}{\fillcircled{1}}
\DeclareUnicodeCharacter{2777}{\fillcircled{2}}
\DeclareUnicodeCharacter{2778}{\fillcircled{3}}
\DeclareUnicodeCharacter{2779}{\fillcircled{4}}
\DeclareUnicodeCharacter{2780}{\fillcircled{5}}
\DeclareUnicodeCharacter{2781}{\fillcircled{6}}
\DeclareUnicodeCharacter{2782}{\fillcircled{7}}
\DeclareUnicodeCharacter{2783}{\fillcircled{8}}
\DeclareUnicodeCharacter{2784}{\fillcircled{9}}
\DeclareUnicodeCharacter{215}{\ensuremath{\times}}
\DeclareUnicodeCharacter{2192}{\ensuremath{\rightarrow}}
\DeclareUnicodeCharacter{394}{\ensuremath{\Delta}}

\newcommand*\floor[1]{\left\lfloor{#1}\right\rfloor}
\newcommand*\ceil[1]{\left\lceil{#1}\right\rceil}

\newcommand*{\Vdd}{\ensuremath{V_{DD}}\xspace}
\newcommand*{\GND}{\ensuremath{GND}\xspace}
\newcommand*{\halfVdd}{\ensuremath{\frac{1}{2}\Vdd}\xspace}
\newcommand*{\tRC}{\ensuremath{t_{RC}}\xspace}
\newcommand*{\tREFW}{\ensuremath{t_{REFW}}\xspace}
\newcommand*{\tREFI}{\ensuremath{t_{REFI}}\xspace}
\newcommand*{\tRFC}{\ensuremath{t_{RFC}}\xspace}
\newcommand*{\tRRDS}{\ensuremath{t_{RRD\_S}}\xspace}
\newcommand*{\tRRDL}{\ensuremath{t_{RRD\_L}}\xspace}
\newcommand*{\tFAW}{\ensuremath{t_{FAW}}\xspace}
\newcommand*{\TRH}{\ensuremath{T_{RH}}\xspace}
\newcommand*{\Nbank}{\ensuremath{N_{bank}}\xspace}
\newcommand*{\HCfirst}{\ensuremath{HC_{first}}\xspace}
\newcommand*{\ie}{\textit{i.e.}\xspace}
\newcommand*{\eg}{\textit{e.g.}\xspace}
\newcommand*{\electron}{{\fontsize{3pt}{3pt} \textbf{\fillcircled{$-$}}}}
\newcommand*{\bigElectron}{{\small \textbf{\fillcircled{$-$}}}}

\newcommand*{\hex}[1]{\ensuremath{\text{\texttt{#1}}_\text{h}}\xspace}
\newcommand*{\bin}[1]{\ensuremath{\text{\texttt{#1}}_\text{2}}\xspace}


\newcommand{\PreserveBackslash}[1]{\let\temp=\\#1\let\\=\temp}
\newcolumntype{C}[1]{>{\PreserveBackslash\centering}p{#1}}
\newcolumntype{R}[1]{>{\PreserveBackslash\raggedleft}p{#1}}
\newcolumntype{L}[1]{>{\PreserveBackslash\raggedright}p{#1}}

\newcommand{\latexComment}[1]{}

\newcommand{\head}[1]{{\noindent\textbf{#1.}\xspace}} % for heading of a paragraph

\newcommand\mycommfont[1]{\small\textcolor{blue}{#1}}
%\SetCommentSty{mycommfont}


\hyphenation{
    aca-demia
    Row-hammer
    Program-mable
}