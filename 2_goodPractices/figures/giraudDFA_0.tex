% !TeX root = ../0_Manuscript.tex
%    trim={left lower right upper}
%\begin{figure}[H]
%    \centering
%%    \includegraphics[width=0.5\textwidth,
%%    trim={20cm 0cm 0cm 0cm},
%%    center]{2_goodPractices/figures/aesFastImpGnd.pdf}
%    \includegraphics[width=0.8\textwidth, center,
%    trim={14.9cm 0cm 0cm 0cm}, clip]{2_goodPractices/figures/aesFastImpGnd.pdf}
%    \caption{Fault susceptibility map analysis}
%    \label{fig:giraudFSM}
%\end{figure}

\begin{figure}[ht]
    \centering
    \begin{subfigure}{0.47\textwidth}
        \includegraphics[width=\textwidth, center,
        trim={14.9cm 0cm 0cm 0cm}, clip]{2_goodPractices/figures/aesFastGndOnly.pdf}
        \caption{FAM in a state-of-the-art \bbi platform. 70 \% of the tested locations exhibits an IC crash, and 15 \% exhibits multibytes-mutibits faults. No Giraud's criterion compatible fault is observed.}
        \label{fig:gndFSM}
    \end{subfigure}
    \hfill
    \begin{subfigure}{0.47\textwidth}
        \includegraphics[width=\textwidth, center,
        trim={14.9cm 0cm 0cm 0cm}, clip]{2_goodPractices/figures/aesFastImpGnd.pdf}
        \caption{FAM in an enhanced \bbi platform. 89.9 \% of the tested area shows a correct behavior, 9.9 \% are incompatible with Giraud's criterion. Five locations show single bit faults, potentially useful for the Giraud attack.}
        \label{fig:giraudFSM}
    \end{subfigure}
    \caption{Fault analysis mapping}
    \label{fig:fam}
\end{figure}
