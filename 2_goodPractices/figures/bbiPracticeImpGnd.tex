% !TeX root = ../0_Manuscript.tex

\begin{figure}[H]
    \centering
    \begin{subfigure}{0.45\textwidth}
        \includegraphics[width=1.0\textwidth, center]{2_goodPractices/figures/model2ProbeManuscrit.pdf}
        \caption{Enhanced \bbi platform model. It describes the voltage pulse generator, the \bbi probe, the transmission line, the IC (1 k\textomega\xspace resistor), the original platform grounding (both 150 \textOmega\xspace resistors), and the proposed enhancements highlighted in red. The enhancements are: creating an approximate impedance matching for the generator, and bypassing the poor grounding with low impedance copper wires.}
        \label{fig:bbiPracticeImpGnd}
    \end{subfigure}
    \hspace{0.045\textwidth}
    \begin{subfigure}{0.45\textwidth}
        \includegraphics[width=1.0\textwidth, center]{2_goodPractices/figures/SIMPLE_BBI_2.pdf}
        \caption{Simulation results of the enhanced \bbi model. The waveforms colors match the schematic. In blue is the ideal voltage pulse (-140 V, 10 ns). In green is the effective pulse received by the IC. In orange is the pulse observed at the generator output. Eventually, in purple is the IC ground current. The dotted waveforms are the waveforms observed in Fig. \ref{fig:bbiPracticeBadGndSignals}, for comparison purposes. The most obvious observed improvements concern the set points, which are fully respected, in addition to the drastic ringing reduction, leading to better temporal control.}
        \label{fig:bbiPracticeImpGndSignals}
    \end{subfigure}
    \caption{\bbi platform enhanced electrical model developed for my thesis to quickly evaluate various platform's parameters, alongside the model simulation results.}
    \label{fig:bbiImpGndGlobalFig}
\end{figure}
