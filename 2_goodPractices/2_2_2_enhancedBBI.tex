% !TeX root = ../0_Manuscript.tex

\section{Enhanced BBI platform model and simulation \ddc}
\label{chap:2_goodPractices;sect:enhancedBBIPlatforms}
In this section, I propose platform enhancements over the state-of-the-art \bbi platform previously introduced.
These improvements aim at being low-cost, fast and easy to set up, to represent an interesting addition without drastically increasing the platform's financial cost.
We will then be able to draw conclusions on such improvements thanks to simulation results.

    \subsection{Matching the generator output impedance \ddc}
    \label{chap:2_goodPractices;sect:enhancedBBIPlatforms;subsect:bbiGenImpMatch}
    % !TeX root = ../0_Manuscript.tex

\begin{figure}[H]
    \centering
    \begin{subfigure}{0.45\textwidth}
        \includegraphics[width=1.0\textwidth, center]{2_goodPractices/figures/model2ProbeManuscrit.pdf}
        \caption{Enhanced \bbi platform model. It describes the generator, the \bbi probe, the transmission line, the IC (1 k\textomega\xspace resistor), the default platform grounding (both 150 \textOmega\xspace resistors), and the enhancements highlighted in red, which are: creating an approximate impedance matching for the generator, and bypassing the poor grounding with low impedance copper wires.}
        \label{fig:bbiPracticeImpGnd}
    \end{subfigure}
    \hspace{0.045\textwidth}
    \begin{subfigure}{0.45\textwidth}
        \includegraphics[width=1.0\textwidth, center]{2_goodPractices/figures/SIMPLE_BBI_2.pdf}
        \caption{Simulation results of the enhanced \bbi model. \textcolor{RoyalBlue}{Blue}: ideal voltage pulse (-140 V, 10 ns). \textcolor{OliveGreen}{Green}: effective signal applied on the IC backside. \textcolor{orange}{Orange}: generator output. \textcolor{Purple}{Purple}: IC ground current. The dotted waveforms are those observed in Fig. \ref{fig:bbiPracticeBadGndSignals}. The most obvious observed improvements concern the set points, which are fully respected, in addition to the drastic ringing reduction, leading to better temporal control.}
        \label{fig:bbiPracticeImpGndSignals}
    \end{subfigure}
    \caption{\bbi platform enhanced electrical model developed for my thesis to quickly evaluate various platform parameters, alongside the model simulation results.}
    \label{fig:bbiImpGndGlobalFig}
\end{figure}

    The first proposed improvement concerns the generated voltage pulse characteristics.
    As we observed previously, the various parameters set points were not met.
    In a fault injection context, it is an undesirable behavior, as it is required to finely control the generated pulse to produce controlled disturbances into ICs.
    Therefore, and because most high speed high voltage pulse generator are specified to be loaded with a precise impedance, I simply propose to connect a known load directly at the output of the generator model.
    In my model's case, a \fiftyOhms{50} resistor is loaded at the generator's output, as illustrated in red in Fig. \ref{fig:bbiPracticeImpGnd}.
    Thus, the generator will see the impedance network formed by the compensation load, the IC, the transmission line, and the grounding installation.
    However, because the grounding installation is platform dependent, it is required, in order to perform a better impedance matching of the generator output, to improve the grounding, which leads us to the following section.

    \subsection{Improving the grounding installation \ddc}
    \label{chap:2_goodPractices;sect:enhancedBBIPlatforms;subsect:bbiGndBetter}
    In many platforms, the grounding installation might be perfectly fine, and the following section may not apply to them.
    However, with our platform, we quickly observed that the grounding impedance was far from negligible.
    Indeed, with an average IC impedance around 1 k\textOmega\xspace, and inter-equipment ground impedance around 150 \textOmega\xspace, it represents a 15 \% increase in the total impedance seen by the generator.
    Therefore, in order to transfer the maximum amount of energy into the IC, especially in areas where the IC impedance might be closer to the grounding impedance, it is required to cancel as much as possible the latter.

    To that end, I propose a very simple setup modification.
    It consists in keeping the platform as is, and adding short copper wires between equipment grounds.
    Therefore, it shunts the platform ground and creates a low-impedance path for electric charges, thus allowing the previous section approximate impedance matching to perform better.

    \subsection{Simulation results \ddc}
    \label{chap:2_goodPractices;sect:enhancedBBIPlatforms;subsect:simRes}
    To verify the soundness of the previously proposed enhancements, I performed simulations thanks to the model presented in Fig. \ref{fig:bbiPracticeImpGnd}, and the simulation results are shown in Fig. \ref{fig:bbiPracticeImpGndSignals}.

    In that case, unlike in the state-of-the-art platform, the voltage set point is almost met concerning the received pulse (green waveform), with a slight undershoot of 6\%.
    It is mirrored on the IC ground current waveform, where the ringing is drastically reduced, which leads to a steeper and more accurate pulse.
    It is especially noticeable when directly comparing the state-of-the-art waveforms in dotted lines.
    Concerning the generator pulse (orange waveform), it is still distorted as the ringing has not disappeared, but is less of a concern since the waveform of interest is the one effectively applied to the IC backside.
        \subsubsection{Load dependency \ddc}
        \label{chap:2_goodPractices;sect:enhancedBBIPlatforms;subsect:simRes;subsubsect:loadDep}
        To further analyze the benefits of the proposed improvements, I performed, as for the state-of-the-art platform, additional simulations with various loads.
        As before, 250 \textOmega\xspace and 2 k\textOmega\xspace were chosen to have a common point of comparison.
        
\begin{figure}[ht]
    \centering
    \begin{subfigure}{0.48\textwidth}
        \centering
        \includegraphics[width=1.0\textwidth, center]{2_goodPractices/figures/SIMPLE_BBI_2_0_IC250.pdf}
        \caption{Enhanced \bbi platform simulation results with an IC load equals to 250 \textOmega\xspace.}
        \label{fig:bbiPracticeImpGndICLoad0}
    \end{subfigure}
    \hfill
    \begin{subfigure}{0.48\textwidth}
        \centering
        \includegraphics[width=1.0\textwidth, center]{2_goodPractices/figures/SIMPLE_BBI_2_0_IC2k.pdf}
        \caption{Enhanced \bbi platform simulation results with an IC load equals to 2 k\textOmega\xspace.}
        \label{fig:bbiPracticeImpGndICLoad1}
    \end{subfigure}
    \caption{Enhanced \bbi platform simulation results with different IC load values.}
    \label{fig:bbiImpGndIcLoadVar}
\end{figure}

        Fig. \ref{fig:bbiImpGndIcLoadVar} presents the simulation results for such loads.
        For both loads, the measured voltage moves away from the set point by around 7 \% in each case.
        It represents a 14 \% range around the -140 V set point, which is immensely better than in the previous platform.
        It is still not perfect, but the platform is overall less dependent to the IC load, which is desirable in order to have repeatable voltage pulse across the entire IC backside.

        Then, quite naturally, for the 250 \textOmega\xspace load, the current is higher than for the 1 k\textOmega\xspace, and with the 2 k\textOmega\xspace, the current is lower.
        In addition to that, thanks to the 50 \textOmega\xspace resistor placed at the generator's output, it reduces the range in which the effective load (the compensation load in parallel with the IC) changes.
        Indeed, it goes from around 42 \textOmega\xspace to about 49 \textOmega\xspace, instead of going from 250 \textOmega\xspace to 2 k\textOmega\xspace in the previous case.
        Eventually, in addition to all of the above, these enhancements have also drastically reduced ringing, which contributes to the applied pulse amplitude being closer to the set point.

    \subsection{Simulation conclusions \ddc}
    All of this leads to better control over the various platform parameters, allowing for more accurate and shorter pulses, closer to the expectations.
    In addition to that, the platform is less design dependent thanks to the minimization of impedance mismatch and poor grounding installation.
    It leads to a better time accuracy, leading to a potentially more controllable fault injection.
%         the 250 \textOmega\xspace load
%        Concerning the 250 \textOmega\xspace load results, displayed in Fig. \ref{fig:bbiPracticeImpGndICLoad0}.

\section{Actual enhanced BBI platform \ddc}
The previous models being a useful tool to draw quick conclusions and predictions, it does not represent the reality.
To that end, I set up the various presented enhancements in an actual \bbi platform in order to verify the soundness of all the outcomes.
