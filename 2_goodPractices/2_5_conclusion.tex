% !TeX root = ../0_Manuscript.tex

\section{Conclusion \ddcu}
\label{chap:2_goodPractices;sect:conclusion}
In this chapter, I first introduced existing \bbi platforms, both in the state-of-the-art and commercially available.
I have shown that multiples solutions ranging from low prices to very high prices exist, each one having its advantages and disadvantages concerning their characteristics.
Thanks to this platform overview, I was able to enumerate the fundamental building blocks of a typical \bbi platform.
Subsequently, I presented the platform used during my thesis experiments, from the custom probe to the generator.
Afterward, I introduced electrical models I designed to quickly compare and evaluate \bbi platforms.
I studied the simulation results of such models, which allowed me to introduce enhancements to existing \bbi platforms, allowing for better accuracy and reproducibility.
Thereafter, I presented experiments performed to verify the soundness of such models, comparing state-of-the-art platforms to the enhanced platform I propose.
Eventually, to go further in the model validation, I described and performed a constraining differential fault attack on a hardware AES coprocessor, sustaining the usefulness of the proposed enhancements.

%In this chapter, we first discussed the hardware and software commonly used for the practice of BBI, as well as presenting our hardware and software.
%We then proposed an enhanced platform for the practice of BBI, consisting of reducing the ground impedance and approximating the voltage pulse generator impedance matching.
%We first studied these enhancements using a coarse platform.
%Furthermore, we then performed analog experiments using real hardware.
%In order to further verify the soundness of the BBI platform improvements, we set up and conducted a Giraud's differential fault attack.
%We observed that it would be impossible to conduct the attack without the aforementioned BBI platform enhancements, thus confirming their usefulness.
