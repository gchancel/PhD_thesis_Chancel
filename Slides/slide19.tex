
\begin{frame}
    %    [trim={left bottom right top},clip]
    \frametitle{Simulation results: Triple-Well}
    \begin{columns}
        \begin{column}{0.3\textwidth}
            \vspace{-1mm}
            \begin{figure}
                \includegraphics[width=\textwidth]{vddMgndManuscritTw.pdf}
                \caption{\centering PDN voltage distribution}
            \end{figure}
            \vspace{-2mm}
            \begin{figure}
                \includegraphics[width=\textwidth]{iEpiTripleWellManuscrit.pdf}
                \caption{\centering Epitaxy current distribution}
            \end{figure}
        \end{column}

        \begin{column}{0.6\textwidth}
            \vspace{-2mm}
            \begin{figure}
                \includegraphics[width=\textwidth]{iSubcManuscritTws.pdf}
                \caption{Substrate current distribution}
            \end{figure}
            \vspace{-1mm}
            \begin{figure}
                \includegraphics[trim={0mm 0mm 0mm 13mm}, clip, width=\textwidth]{CurrentDistributionNormT140eT10_ExcTw.pdf}
                \caption{Per-layer normalized substrate current density: }
            \end{figure}
            \centering 186 µm diameter mid-height
        \end{column}
    \end{columns}
\end{frame}
\begin{frame}{Dual-well vs Triple-well}
    Differences between Dual-well and Triple-well circuits:
    \begin{itemize}
        \item Dual-well → NMOS are \textcolor{blue}{DC-coupled} | PMOS are \textcolor{orange}{AC-coupled}
        \item Triple-well → entire IC is \textcolor{orange}{AC-coupled}
    \end{itemize}
    Implications:
    \begin{itemize}
        \item Dual-well → continuous energy flow during the pulse
        \item Triple-well → energy flow confined to the pulse edges
    \end{itemize}
    \vspace{6mm}
    \centering \textbf{For a given amount of time → less energy injected into TW compared to DW\\
    \hspace{10.54em} → less RMS current into TW compared to DW}
\end{frame}
