\algdef{SE}% flags used internally to indicate we're defining a new block statement
[CLASS]% new block type, not to be confused with loops or if-statements
{Class}% "Class{name}" will indicate the start of the struct declaration
{EndClass}% "EndClass" ends the block indent
[1]% There is one argument, which is the name of the class
{\textbf{class} \textsc{#1}}% typesetting of the start of a struct
{\textbf{end class}}% typesetting the end of the struct

\algdef{SE}% flags used internally to indicate we're defining a new block statement
[METHOD]% new block type, not to be confused with loops or if-statements
{Method}% "Method{name}" will indicate the start of the struct declaration
{EndMethod}% "EndMethod" ends the block indent
[2]% There is one argument, which is the name of the data structure
{\textbf{method} \textsc{#1}}% typesetting of the start of a struct
{\textbf{end method}}% typesetting the end of the struct

%\begin{figure}[ht]
%    \begin{Verbatim}[frame=single]
%        .subckt elementary_bloc D F L R Re U
%        R1 U N001 RH
%        R2 N001 D RH
%        R3 Re N001 RL
%        R4 N001 F RL
%        R5 N001 L RL
%        R6 R N001 RL
%        .ends elementary_bloc
%    \end{Verbatim}
%    \caption{Elementary substrate SPICE netlist}
%    \label{fig:subSpiceNetlist}
%\end{figure}

\begin{figure}[ht]
    \begin{Verbatim}[frame=single]
    .subckt elementary_blocx6 D1 D2 D3 D4 D5 D6
    +F1 F2 F3 F4 F5 F6 L R RE1 RE2 RE3 RE4 RE5 RE6
    +U1 U2 U3 U4 U5 U6 VSUBCintC
    XX1 D1 F1 L VSUBCintL2 RE1 U1 elementary_bloc
    XX2 D2 F2 VSUBCintL2 VSUBCintL1 RE2 U2 elementary_bloc
    XX3 D3 F3 VSUBCintL1 VSUBCintC RE3 U3 elementary_bloc
    XX4 D4 F4 VSUBCintC VSUBCintR1 RE4 U4 elementary_bloc
    XX5 D5 F5 VSUBCintR1 VSUBCintR2 RE5 U5 elementary_bloc
    XX6 D6 F6 VSUBCintR2 R RE6 U6 elementary_bloc
    .ends elementary_blocx6
    \end{Verbatim}
    \caption{SCS substrate layer SPICE netlist}
    \label{fig:subSpiceSCS}
\end{figure}

\begin{algorithm}[ht]
\caption{Integrated circuit SPICE netlist generation algorithm.}
\label{alg:icGen}
    \begin{algorithmic}
        \Require SUBTYPE \Comment{IC substrate type: Dual-well, Triple-well, Mixed}
        \Require TSUB \Comment{IC substrate thickness}
        \Require ESUB \Comment{Elementary substrate block thickness}
        \Require VPUU \Comment{Voltage pulse amplitude}
        \Require PW \Comment{Voltage pulse width}
        \Require TFR \Comment{Voltage pulse rise and fall times}
        \Require SIMTIME \Comment{Simulation duration}
        \Require SIMSTEP \Comment{Simulation time step}
        \Require TEX \Comment{Desired X size (µm)}
        \Require TEY \Comment{Desired Y size (µm)}
        \Require prbX \Comment{BBI probe X coordinate}
        \Require prbY \Comment{BBI probe Y coordinate}
        \State $RH \gets 2000$ \Comment{Elementary substrate up-down/front-rear resistor value}
        \State $RL \gets 500$ \Comment{Elementary substrate left-right resistor value}
        \State $WSEG \gets 30$
        \State $HSEG \gets 5$
        \State $W6SEG \gets 30 \div 6$
        \State $nC \gets TEX \div WSEG$ \Comment{Number of column}
        \State $nL \gets TEY \div HSEG$ \Comment{Number of lines}
        \State $nH \gets TSUB \div ESUB$ \Comment{Number of substrate layers}
        \Ensure nC, nL and nH are integers
        \State \textbf{var} SCS[nL, nH] \Comment{Array containing each standard-cell properties}
    %    \State $RH, RL \gets \Call{subRHcalc}{ESUB, RH, RL}$
        \State $RH \gets RH \times (ESUB \div 10)$ \Comment{Adjust RH value according to user defined variable}
        \State $RL \gets RL \times (ESUB \div 10)$ \Comment{Adjust RL value according to user defined variable}
        \ForAll{cY in \textlbrackdbl 0 ; nL\textlbrackdbl}
            \ForAll{cX in \textlbrackdbl 0 ; nH\textlbrackdbl}
                \State $\overrightarrow{X} \gets
                \left[\begin{array}{r}
                            cX \times WSEG\\
                            cX \times WSEG + 1 \times (W6SEG \div 2)\\
                            cX \times WSEG + 3 \times (W6SEG \div 2)\\
                            cX \times WSEG + 5 \times (W6SEG \div 2)\\
                            cX \times WSEG + 7 \times (W6SEG \div 2)\\
                            cX \times WSEG + 9 \times (W6SEG \div 2)\\
                            cX \times WSEG + 11 \times (W6SEG \div 2)\\
                            cX \times WSEG + 12 \times (W6SEG \div 2)\\
                    \end{array}\right]$
                $;\overrightarrow{Y} \gets
                \left[\begin{array}{r}
                            cY \times HSEG\\
                            (cY + \frac{1}{2}) \times HSEG\\
                            (cY + 1) \times HSEG\\
                    \end{array}\right]$
                \If{$\overrightarrow{Y}[0] = 0 \lor \overrightarrow{Y}[0] = TEY$} \Comment{Determines if SCS has external power}
                    \State SCS[cY, cX].power = True
                \Else
                    \State SCS[cY, cX].power = False
                \EndIf
                \If{$\overrightarrow{X}[0] = prbX \land \overrightarrow{X}[2] = prbX \land \overrightarrow{Y}[0] \leqslant (prbY + 15) \land \overrightarrow{Y}[0] \geqslant (prbY - 15)$}
                    \State SCS[cY, cX].probe = True
                \Else
                    \State SCS[cY, cX].probe = False
                \EndIf
            \EndFor
        \EndFor
    \end{algorithmic}
\end{algorithm}
