\section{Preliminary models validation \ddcnew}
Designing and creating the models I previously described is the first step in practically implementing them.
Afterward comes various validation to verify the soundness of such models.
To that end, using the generation algorithm, I created various ICs with the following measurements: a width of $550 \; \mu m$, a depth of $450 \; \mu m$, and a substrate thickness of $140 \; \mu m$.
It is, according to our platform computational power, a reasonable size/calculation time ratio.
Three ICs are generated:
\begin{itemize}
    \setlength\itemsep{-0.1em}
    \item An exclusive \dwF circuit, to isolate the \dwF specific \bbi effects with further simulations;
    \item An exclusive \twF circuit, to isolate the \twF specific effects under \bbi;
    \item Eventually, a mixed substrate IC, comprising at the same time \dwF and \twF \scs, to mimic a real IC, more specifically our platform microcontroller.
\end{itemize}
For each preliminary validation, the IC is simply powered thanks to its power inputs, and no \bbi probe is connected to any of them.

Then, for each generated IC, I defined validation criteria, consisting in:
\begin{itemize}
    \setlength\itemsep{-0.1em}
    \item Measuring the global IC quiescent leakage current to verify any inconsistencies. Indeed, a steady should draw a reasonable amount of power. Therefore, verifying this criterion allows verifying if the model is flawed, specifically concerning the substrate blocks interconnections and the substrate connection to toe top of the \scs;
    \item Measuring the power network IR drop. It is complementary to the previous measurement and allows spotting any inconsistencies in the power delivery network design or generation.
\end{itemize}
