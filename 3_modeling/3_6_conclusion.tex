\section{Conclusion \ddcu}
In this chapter, I have introduced a modeling and simulation flow for Body Biasing Injection.
The models are based on a typical integrated circuit structure, including the power delivery network, the logic gates and the silicon substrate.
They allowed me to simulate various IC structures under \bbi.
The most important things they taught me are how energy is flowing inside the substrate to reach the logic gates.
Thanks to an analysis of charge concentration in the substrate, I was able to observe that most of the energy flows below the \bbi probe.
Therefore, it shows that \bbi seems to be a local fault injection method, such as \lfi, even if their precision is on different levels.
Then, thanks to an in-depth study of two substrate types: \dwF and \twF, I was able to draw conclusions on how \bbi pulses impact the energy propagation inside ICs.
I observed that for negative pulses, in \dwF substrates, ICs are DC-coupled with the probe, thus allowing for energy to flow during the entire duration of the pulse, which leads to more effective energy being transferred into the IC.
Therefore, it represents a potentially higher hazard concerning IC integrity.
On the other hand, in \twF substrates and negative pulses, ICs are entirely AC-coupled to the probe, thus allowing energy to only flow on the edges of the pulse.
It means that less effective energy is transferred, leading to potentially less dangerous experiments.

Then, I explained why I only considered negative pulses for my work.
Because the pulse polarity has an impact on the coupling, and because positive pulses allow for more resistive paths for the charges, it is potentially more dangerous for the IC target.
These difference in IC integrity hazard have been observed in actual experiments.
Indeed, performing a single \gcfam such as it was presented in the second chapter led the target to cease to function.
However, I did not conduct extensive testings concerning these hazards in an objective to preserve as much IC targets as possible for the success of my experiments.
%These differences mean that for a given disturbance, a lower amount of energy can be transferred into a \twF IC compared to a \dwF IC.
%Therefore, it is important to keep this in mind concerning the potential hazards \bbi practice on \dwF ICs could involve.

%In this chapter, I have introduced a modeling and simulation flow for Body Biasing Injection.
%The models are based on a typical integrated circuit structure, including the power delivery network, the logic gates and the silicon substrate.
%They consist of the elementary building blocks called Standard-Cell Segments, mostly used in digital IC design, which are a set of multiple logic gates assembled together to form a dedicated logic function.
%I started the modeling thanks to M. Dumont work on \emfi modeling and simulation \cite{mathieuEMFI}.
%Then, I improved and adapted these models to a \bbi context, by improving the silicon substrate sub models to fully appreciate its role during \bbi.
%In addition to this, I considered two types of substrate types commonly used in ICs: \dwF and \twF.
%This allowed me to analyze the key differences between those substrate types under \bbi.
%Then, I introduced the importance of the voltage pulse generator model, which, depending on the architecture of the actual generator, can have an impact on simulation results.
%Eventually, I presented various validation for the models, to verify their soundness and potential inconsistencies.




%In this chapter, we presented enhanced electrical models which can be utilized to simulate integrated circuits under body biasing fault injection.
%These models, supported by older ones originally designed for ICs under EMFI, cover two substrate types commonly found in commercial ICs: dual-well and triple-well substrates.
%The substrate type is of great importance when considering BBI as it is the only physical environment where charges can circulate.
%Each sub-models contain:
%\begin{itemize}
%    \item The power delivery network
%    \item The average electrical model of a hundred of logic gates
%    \item The various silicon junctions
%    \item The silicon substrate
%\end{itemize}
%
%Standard-cells segments models representing a portion of an IC, they need to be replicated and connected with each other in order to be meaningful.
%In addition to this, they propose refined substrate sub-models in order to improve the model spatial accuracy over their predecessors.
%The main advantage of these models is their relative lightness, computationally speaking.
%Indeed, they are only composed of passives components, in order to be able to simulate large resulting ICs.
%However, their main advantage is also their main shortcoming, they do not represent any function of the modeled IC, but its average electrical behavior.