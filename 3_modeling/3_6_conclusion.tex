\section{Conclusion \ddcu}
In this chapter, I have introduced a modeling and simulation flow for Body Biasing Injection.
The models are based on a typical integrated circuit structure, including the power delivery network, the logic gates and the silicon substrate.
They consist of the elementary building blocks called Standard-Cell Segments, mostly used in digital IC design, which are a set of multiple logic gates assembled together to form a dedicated logic function.
I started the modeling thanks to M. Dumont work on \emfi modeling and simulation \cite{mathieuEMFI}.
Then, I improved and adapted these models to a \bbi context, by improving the silicon substrate sub models to fully appreciate its role during \bbi.
In addition to this, I considered two types of substrate types commonly used in ICs: \dwF and \twF.
This allowed me to analyze the key differences between those substrate types under \bbi.
Then, I introduced the importance of the voltage pulse generator model, which, depending on the architecture of the actual generator, can have an impact on simulation results.
Eventually, I presented various validation for the models, to verify their soundness and potential inconsistencies.

%In this chapter, we presented enhanced electrical models which can be utilized to simulate integrated circuits under body biasing fault injection.
%These models, supported by older ones originally designed for ICs under EMFI, cover two substrate types commonly found in commercial ICs: dual-well and triple-well substrates.
%The substrate type is of great importance when considering BBI as it is the only physical environment where charges can circulate.
%Each sub-models contain:
%\begin{itemize}
%    \item The power delivery network
%    \item The average electrical model of a hundred of logic gates
%    \item The various silicon junctions
%    \item The silicon substrate
%\end{itemize}
%
%Standard-cells segments models representing a portion of an IC, they need to be replicated and connected with each other in order to be meaningful.
%In addition to this, they propose refined substrate sub-models in order to improve the model spatial accuracy over their predecessors.
%The main advantage of these models is their relative lightness, computationally speaking.
%Indeed, they are only composed of passives components, in order to be able to simulate large resulting ICs.
%However, their main advantage is also their main shortcoming, they do not represent any function of the modeled IC, but its average electrical behavior.