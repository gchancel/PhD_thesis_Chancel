\section{Conclusion \ddcold}

In this chapter, we presented enhanced electrical models which can be utilized to simulate integrated circuits under body biasing fault injection.
These models, supported by older ones originally designed for ICs under EMFI, cover two substrate types commonly found in commercial ICs: dual-well and triple-well substrates.
The substrate type is of great importance when considering BBI as it is the only physical environment where charges can circulate.
Each sub-models contain:
\begin{itemize}
    \item The power delivery network
    \item The average electrical model of a hundred of logic gates
    \item The various silicon junctions
    \item The silicon substrate
\end{itemize}

Standard-cells segments models representing a portion of an IC, they need to be replicated and connected with each other in order to be meaningful.
In addition to this, they propose refined substrate sub-models in order to improve the model spatial accuracy over their predecessors.
The main advantage of these models is their relative lightness, computationally speaking.
Indeed, they are only composed of passives components, in order to be able to simulate large resulting ICs.
However, their main advantage is also their main shortcoming, they do not represent any function of the modeled IC, but its average electrical behavior.