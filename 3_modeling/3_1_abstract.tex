
\section{Summary \ddcu}
%This chapter presents the work carried out concerning the modeling and simulation of integrated circuits and platforms in a body biasing fault injection context.
%The presented work focused on elaborating electrical models allowing to evaluate with simulations the behaviors of ICs subjected to BBI.
%The chapter introduces the elaborated models and the algorithms used to create them, and then goes on to present various validation steps to check the meaningfulness of the models.

%This chapter introduces and develop the work carried out concerning the integrated circuit modeling in a \bbi context.
%It begins with an introduction of the various aspects on how IC are manufactured, including power delivery networks and silicon substrate types, depicting the main aspects of their structure.
%Afterward, it introduces electrical models allowing to simulate integrated circuits under \bbi.
%Then, it lingers on how to properly model the voltage pulse generator and the electrical probe, which are the main tools for performing \bbi.
%Eventually, it shows the study of how actual logic gates react to \bbi disturbances and the implications of such results.
%Parts of this work have been published both in \cite{mybbiCosade} and \cite{mybbiFdtc2022}.

This chapter is dedicated to introducing the work I carried out concerning the modeling and simulation of integrated circuits, subject to \bbi.
It begins with a discussion of the various aspects involving how ICs are designed and manufactured.
It includes a thorough description of their power delivery network and silicon substrate.
The main aspects of their structure, being inherited from the standard design flow provided by CAD vendors, are also described.
Afterward, it introduces electrical models allowing to simulate integrated circuits under \bbi.
Then, it lingers how to properly model the voltage pulse generator and the electrical probe, which are the main tools for performing \bbi.
Eventually, it shows the study of how actual logic gates react to \bbi disturbances and the implications of such results.
Parts of this work have been published both in \cite{mybbiCosade} and \cite{mybbiFdtc2022}.
