% !TeX root = ../0_Manuscript.tex

\chapter{Conclusion and perspectives}
\label{chap:6conclusion}
\vspace{-3cm}
\minitoc
\newpage

%\section{General conclusion}
Nowadays, fault injection is a major topic when studying hardware security.
Indeed, in the past three decades, numerous new fault injection methods have emerged,
Among them, we can distinguish electromagnetic fault injection, relying on, as its name implies, the creation of parasitic electric current with the help of electromagnetic probes.
It provides a good efficiency while keeping the cost low.
Then, there is laser fault injection, a very effective and efficient fault injection technique relying on the photoelectric effect at work in silicon transistors.
However, it is an expensive technique when compared to others, and requires great mastery of the equipment with acute knowledge.
Eventually, and not to cite them all, body biasing injection was created at the beginning of the years 2010 in my lab.
More recent than the others, it is also a less studied technique.
However, in my thesis, I brought my contributions to this technique.
%
% as technology nodes advance, and as new fault injection methods emerge, it is required to evaluate deeply their behavior on secure integrated circuits.


\section{Contributions}
\label{concl_contrib}

    \subsection{Improvements of existing \bbi platforms}
    Because \bbi was not a very much documented fault injection method before my thesis, the platforms as a whole were not optimized.
    Indeed, most experiments were difficult to reproduce without major variability.
    In addition to this, with our original platform, the range between a normal behavior and a crash of the target was very tight, therefore preventing the execution of fault attacks.
    The goal of circumventing these limitations allowed me to set up enhancements for the practice of \bbi, at the same time efficient and easy to set up.

    \subsection{\bbi simulation flow}
    To better understand the mechanisms at work in integrated circuits under body biasing injection, I developed a simulation flow, allowing me to predict the various disturbances appearing inside the IC targets.
    The base models were extracted from previous works concerning \emfi simulation, and they were enhanced to match the needs of \bbi simulation.
    The models were developed for bulk technologies, considering both \dwF and \twF substrate types.
    However, these models were made to allow for fast simulation of relatively large circuits.
    Therefore, they did not include logic functions.
    To that end, I extended the simulation flow to properly consider logic gates' behavior under \bbi.
    Eventually, these analyses allowed me to understand the mechanisms at work in fault creation under \bbi.

    \subsection{First documented differential fault attack using Giraud's criterion with \bbi}
    Thanks to the platform enhancements, I was able to observe repeatably single bit faults, which are useful when considering constraining fault models.
    Creating single bit faults can be beneficial because they are often used in powerful fault attacks, such as the single bit Giraud's differential fault attack.
    Therefore, by using single bit faults data obtained on my target AES coprocessor, I was able to conduct a Giraud's \dfa and retrieve 14 out of 16 bytes of the secret AES key.

    \subsection{Substrate thinning analysis and \bbi efficiency}
    Similar to what has been done in the past for \lfi \cite{lfiThinning, lfiThinning2}, I conducted similar research on \bbi concerning IC substrate thinning.
    Because the substrate is the physical environment conveying the charge carriers, analyzing what happens, both with electric models and with actual circuits, when the substrate thickness changes, is fundamental.
    I observed that thinning the substrate is mainly useful when the generator power is not sufficient.
    Therefore, while there is enough generator power, increasing the voltage with the substrate thickness leads to very similar results.


\section{Outlooks and discussions}
\label{concl_outlook}
While I have brought new knowledge concerning body biasing injection, there is a lot of room for improvement and for additional work to be done.

First, as I have mentioned in the second chapter, better impedance matching methods can be implemented.
The simplest one is depicted in Fig. \ref{fig:impMatchPhotoNew}.
We started using this implementation in the lab, and it has proven to be more efficient at matching the generator output impedance by providing less ringing and better voltage and pulse width set point values.
Then, to even better provide impedance matching, an active system should be implemented, therefore allowing for adaptability over various pulse generators.

Then, many other aspects can be explored using \bbi.
For instance, RAM fault injection has not yet be studied, and similar to what has been done with \lfi \cite{lfiFaultModel}, memory elements behavior, both static and dynamic, under body biasing injection, could help to have a more profound understanding of the mechanisms at work inside the integrated circuits.
Afterward, FLASH memory modifications, if possible, should be investigated, as if such data modification is possible using \bbi, it could be compelling.

Concerning the modeling of \bbi, especially the simulation follow-up presented in the fourth chapter, more in-depth analysis of more complex logic IC has to be conducted to fully verify the soundness of my conclusions.
For example, and as the substrate simulation is one of the heaviest calculations, computationally speaking, when performing \scs simulation, the two steps simulation flow I presented could be maintained, while simulation an equivalent logical circuit analogous to the \scs one.
However, if by doing so, it eliminates the simultaneous calculation of substrate behavior and logic gates behavior, it still requires a lot of computational power.
Nevertheless, a single \scs simulation data could be used in many logic gates simulations, thus reducing the total computing time.
Eventually, it could be interesting to further explore the geometric approach, especially when considering the various areas of an IC which I did not study, such as the FLASH or the RAM, and analyzing if the geometric approach still makes sense.


\section{Final remarks}
\label{concl_finalremark}
