% !TeX root = ../0_Manuscript.tex

%\section*{\Huge General introduction \ddc}
%\sectionmark{General introduction}
%\textbf{\Huge General introduction \ddc}
%\thispagestyle{}
%\section{General introduction \ddc}
%\textcolor{orange}{IN FIRST HERE WILL BE THE GENERAL INTRO* ABOUT SECURITY, FAULT INJECTION AND CRYPTO* IN ICs NOWADAYS.}
Over the past twelve years, various fault injection methods have been extensively studied.
The most noteworthy were Electromagnetic Fault Injection (EMFI) and Laser Fault Injection (LFI).
Indeed,among all studies, elaborated models have been proposed to study and predict the effects of EMFI on integrated circuits (IC), and IC substrate thinning effects have been studied concerning LFI efficiency.
However, Body Biasing Injection (BBI), although introduced in 2011, has been less documented than the above injection methods.
Within this context, this work aims at tackling the interests of this technique over others, replacing them or

The \cRef{chap:1_stateOfTheArt}{first chapter} of this manuscript presents the global fault injection and specific Body Biasing Injection state of the art, mainly concerning side-channel attacks.
Various fault injection platforms are presented, ranging from electromagnetic fault injection to laser-fault injection, eventually introducing body-biasing injection.

Then, the \cRef{chap:2_goodPractices}{second chapter} introduces new enhanced practices for body biasing injection.
It aims at presenting the work concerning various improvements for the practice of BBI.
These contributions aim, thanks to minor modifications and improvements of existing platforms, at improving body biasing injection efficiency.

Afterward, the \cRef{chap:3icModeling}{third chapter} focuses on IC modeling specifically for the practice of BBI.
It introduces electrical models and algorithms allowing to generate and simulate integrated circuits in a BBI context.
The introduced models have the advantage to offer simulation duration on a human timescale, thus allowing to evaluate and study large circuits in short amount of time.

Subsequently, the \cRef{chap:4thinning}{fourth chapter} discusses a common practice when performing fault injection: the thinning of integrated circuits' substrate.
This topic has been addressed extensively concerning laser fault injection, and we present our contribution concerning BBI.
It mainly relates to studying IC behavioral differences and BBI efficiency.
Various models are proposed in order to get different approaches of the subject, allowing to predict differently electrical and physical phenomenons.
Mathematical models are also derived from the previous models, allowing to calculate optimal experimental parameters in addition to predicting circuit behavior.

Eventually, the \cRef{chap:5faultModel}{fifth and last chapter} introduces a fault model, allowing to explain at a circuit level and at a transistor level how faults are created under body biasing injection.

%\clearpage
%
%\pagestyle{default}
