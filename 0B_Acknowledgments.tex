% !TeX root = ./0_Manuscript.tex

\section*{Acknowledgements \DDC}
Before getting started, I would like to express my gratitude to all the people who have given me their support, whether professional, personal and emotional, and made this thesis possible.

Formerly, I wish to thank my thesis director, Philippe Maurine, and my thesis supervisor, Jean-Marc Gallière, for their guidance, patience, and support throughout these three years.
Thanks to their advice, knowledge and their constant assistance when I was in need, I have succeeded in doing my work properly.
First, I thank Philippe Maurine for its scientific support and teaching of the various new fields for me at my arrival, while being understanding, in addition to the regular meetings and discussions which helped me to stay on course throughout these three years.
Second, I thank Jean-Marc Gallière for his expertise on the various aspects of integrated circuits modeling and simulation, in addition to his understanding and listening.

I would also like to thank Laurent Latorre, director of our lab microelectronics department and a former teacher of mine, and Éric Dubreuil, also a former teacher of mine, for allowing me to have the chance to teach and transmit, twice during my thesis, to students and future engineers my passion: electronics.

Then, I thank Richard Vareilhes for his extended help and precious teaching during my first two years in higher education, concerning mathematics, physics, chemistry and control engineering, without whom I would not be here today.

Additionally, I thank Éric Dubreuil for his listening, understanding and advice during difficult times.
I also thank Florent Bruguier, who allowed me to teach the various aspects of physical measurements in a side-channel context to the students of the DE SECNUM of Polytech Montpellier.
I extend my thanks to every former teacher of Polytech Montpellier, who made my thesis possible through their past teachings.

Thereafter, I thank Caroline Lebrun and Cécile Lukasik for their very much appreciated support in difficult times, as they helped me to stay upright.
I also thank Jérémie Salles for his patience and constant support for the tedious and problematic electric simulations, where I often crashed the lab computer, and Laurent De Knyff, for his advice and knowledge, in addition to his professionalism when it came to PCB manufacturing and every aspect related to our lab bench.

During these three years spent at the LIRMM, I have met numerous people from the lab and outside, who helped me get through hard times, who made me laugh when I was in need, with whom I had the chance to share my meals most of the days.
I thank Jonathan Miquel, my former office colleague, Paul Delestrac, Alexis Blaya and Pierre Groc for the laughs and the support.

Now, I would like to thank my closest friends, some I have known before, some during who have supported me.
I thank Dorian Alazard, Xavier Dartis and David Camarazo for, their uniqueness, their kindness, their complicity, despite being most of the time geographically separated, with whom I have had the chance to share moments of joy and rest.
I also thank Matthieu Béchet and Mohammadali Zoroufchian, former classmates and friends, for our shared meals, laughs, support and our various fascinating discussions during all these years.
Then, I deeply thank Julien Toulemont, close friend and colleague, who introduced me to our test platforms, for his humility, for all our coffee breaks, meals, our laughs, our grumpiness at times, allowing us to relax in hard times.
Afterward, I deeply thank Loïc France, with whom we shared meals, laughed, discussed video games, electronics and various other topics, for all he has taught me, for his realism, his thoroughness, his knowledge, and his constant support during my thesis and more.

My heartfelt thanks go to my nearest, dearest, my family.
First, I deeply thank my amazing partner, Olivia Serenelli-Pesin, who I had the chance to meet during the first part of my thesis, who helped, supported, accompanied, brought me joy and happiness, and loved me all this time, in addition to her precious advice, without whom this thesis would not have made it to the end.
Then, I thank profoundly my parents and my brother, who have accompanied me since I was born.
They brought me constant encouragement, especially in times of need.
They have supported me during my studies, in uncertain and doubtful times, and brought me joyfulness and stability all over these years.

Eventually, I wish to thank my two rabbits, Mimolette and Boo, my two hamsters, Phophe and Philippe, and my two gerbils, Moiselle and Boule, who have brought and continue to bring me to this day joy, happiness, calmness, in addition to their innocence and lightness in life, which reminds me of how simple things can be.

%\textit{The authors acknowledge the support of the French Agence Nationale de la Recherche (ANR), under grant ANR-19-CE39-0008 (project ARCHI-SEC).
%They also acknowledge the French Ministère des Armées -- Agence de l’innovation de défense (AID) under grant ID-UM-2019 65 0036.}