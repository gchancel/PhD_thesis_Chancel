% !TeX root = ./0_Manuscript.tex

\section*{Abstract \DDC}
Fault injection techniques have been extensively developed and studied in the past decades.
Various approaches have emerged, using different physical quantities' manipulation to disturb integrated circuits behavior, such as electromagnetic fields to create parasitic currents in the IC targets, also known as electromagnetic fault injection.
Additionally, light emission with the use of laser, leveraging the silicon photoelectric effect, also known as laser fault injection.
Then, simple power supply voltage or clock signal manipulations, enabling to run integrated circuits outside their specifications, thus provoking exploitable unexpected behavior, known as glitch fault injection.
Eventually, biasing the substrate of integrated circuits with the help of fast high-voltage pulses, known as body biasing injection, not to cite them all.
Among these methods, laser fault injection, electromagnetic fault injection or glitch fault injection were extensively studied.
Therefore, countermeasures allowing to protect and secure integrated circuits were proposed for most of them.
However, body biasing injection did not get this attention.
Consequently, no specific countermeasure or understanding of this technique were highlighted.

In this context, my thesis' work brings some answers and insights concerning body biasing injection.
I propose various enhancements of the state-of-the-art way of practicing body biasing injection, thus enabling more repeatable and more reliable experiments.
Subsequently, I demonstrate the feasibility, thanks to body biasing injection, of a differential fault attack relying on a constraining single-bit fault model.
Then, I introduce a new modeling and simulation flow dedicated to the study of body biasing injection.
This methodology, based on previous works on electromagnetic fault injection, allows understanding the effect of body biasing injection on integrated circuits, going from electric charge behavior, to logic gates disturbance and a fault model.
Eventually, I present the study of the interest in thinning the substrate of integrated circuits for the practice of body biasing injection, illustrating the consequences of such a practice on the efficiency of this fault injection method.

\section*{Résumé de la thèse \DDC}
Ces dernières décennies, les méthodes d'injection de fautes ont été étudiées de manière approfondie.
De nombreuses approches ont vu le jour, toutes utilisant diverses grandeurs physiques dans le but de perturber le comportement des circuits intégrés ciblés.
On peut identifier dans un premier temps les champs électromagnétiques permettant de créer des courants parasites dans les circuits, connu sous le nom d'injection de fautes par impulsion électromagnétique.
De plus, les émissions de photons produites grâce au laser, utilisant l'effet photoélectrique du silicium, connues sous le nom d'injection de fautes par impulsion laser.
Ensuite, des méthodes plus élémentaires, tirant parti de la perturbation des signaux d'alimentation ou d'horloge, créent des comportements indésirables exploitables, communément appelés injection de fautes par glitch.
Enfin, on peut identifier la polarisation transitoire du substrat des circuits intégrés, aussi connue sous le nom d'injection de fautes par impulsion dans le substrat.
Parmi ces méthodes, l'injection par impulsions laser, par impulsions électromagnétiques et par glitch ont été étudiées en détail.
En revanche, l'injection de fautes par impulsion dans le substrat n'a pas eu ce traitement.
Par conséquent, aucune contremesure dédiée n'a été proposée pour cette technique.

Dans ce contexte, mon travail de thèse de doctorat vise à apporter des éléments de réponse concernant l'injection de fautes par impulsion dans le substrat.
Dans un premier temps, je propose des améliorations des plateformes préexistantes, permettant de réaliser des expérimentations plus reproductibles et plus fiables.
Ensuite, je démontre la faisabilité d'une attaque par faute différentielle qui s'appuie sur un critère contraignant.
Ultérieurement, je présente une nouvelle méthodologie de modélisation et de simulation de l'injection de fautes par impulsion dans le substrat.
Cette méthode, fondée sur de précédents travaux réalisés concernant l'injection de fautes par impulsion électromagnétique, permet d'avoir une introspection sur les comportements internes d'un circuit soumis à l'injection par impulsion dans le substrat.
Enfin, j'étudie l'intérêt d'affiner le substrat des circuits intégrés destinés à être utilisés avec l'injection par impulsion dans le substrat, ce qui permet d'appréhender les conséquences de cette pratique sur la méthode d'injection de fautes.
