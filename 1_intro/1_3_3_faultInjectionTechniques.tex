\section{Fault-injection techniques \ddc}
\label{chap:1;sect:fInjTech}

\subsection{Glitch fault injection}
\label{chap:1;sect:fInjTech;subsect:glitch}
Glitch fault injection (GFI) are one of the first historical documented fault injection attacks.
They are simple and require little equipment.
For the most part, they are non-invasive, which means that they are reversible, physically speaking.
Various physical quantities can be disturbed, but the power supply voltages (VDD or GND), and the IC clock are the most common.
Each physical quantity can be modified at the attacker's discretion, with a certain amount.
However, the disturbances have to be short enough to avoid IC shutdown concerning power supply glitches, but also not powerful enough to avoid the IC destruction.
On the one hand, the main advantage of such attack is its easiness to set up compared to other methods.
On the other hand, their main disadvantage is the complete lack of locality with the injection effects.
Indeed, disturbing IC's macro-parameters interfere with the entire chip and does not guarantee a useful faulty behavior.
In addition to that, every modern IC is prepared to detect such attacks and thus protect itself by resetting its electronics.

\subsection{Laser fault injection}
\label{chap:1;sect:fInjTech;subsect:lfi}
Laser fault injection (LFI) has been introduced in 2002 \cite{firstLfi} and is a more complex technique than GFI.
However, its precision is immensely better, at the cost of being semi-invasive, and sometimes invasive.
LFI consists in targeting specific regions of the IC with laser beams of specific wavelengths.
Several other parameters are involved for this method to succeed, such as the light emission duration and the area/volume of the targeted region.

\subsection{Electromagnetic fault injection}
\label{chap:1;sect:fInjTech;subsect:emfi}
Electromagnetic fault injection (EMFI) is a more recent and more studied technique, introduced in 2002 \cite{firstEmfi2002}.
Its principle is basic: an electric current in a wire (probe) near an IC creates a corresponding electric current in the IC power delivery network, similar to an electric transformer.
Similar to GFI, the attack can be non-invasive, although this method yields better results while being semi-invasive.
Indeed, the closer the probe to the IC, the better the coupling and the mutual inductance, which often required to remove the IC's plastic package.
