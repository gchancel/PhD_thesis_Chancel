
\section{Summary \ddc}
\label{chap5:sect:summary}
This chapter presents the extended simulation flow based on the one presented in the previous chapter.
Indeed, the \scs models developed and used for \bbi simulation are not sufficient when it comes to explaining the mechanisms at work concerning fault creation.
This is because these models do not include any logic function of the considered ICs.
Their design is a compromise between electrical accuracy and time cost.
Effectively, with the available technology, it is not possible, temporally speaking, to simulate millions of transistors while considering their electric and logic characteristics.
%This chapter is dedicated to extend the simulation flow presented in the previous chapter.
%Indeed, as we have seen, the \scs models developed and used for the simulations do not consider the logic function of the simulated IC.
However, when working with fault injection mechanisms, it is fundamental to appreciate the logical behavior of an IC target to properly understand how faults occur.
This allows developing a fault model and eventually designing countermeasures.
In this context, I extend the previous simulation flow by adding additional steps to it.
%Therefore, and because it is important in a fault injection context to understand how faults occur to be able to develop a fault model, I am going to extend the previous simulation flow to consider logic function behavior under \bbi.

%In this chapter, we introduce a fault model for BBI.
%It aims at explaining how and why faults occur in ICs subject to body biasing injection.
%Thanks to chapter \ref{chap:3icModeling}'s electrical models as a basis, it allows explaining how electrical charges displacement in the IC during a BBI pulse allows changing some logic gates output values.
%Therefore, thanks to the ability to finely control the induced disturbances, it is possible to target critical time in the IC calculation.
%Eventually, to verify the correctness of the proposed analysis, both substrate charges propagation and logic gates behavior studies are conducted.

%In this chapter, we present a fault model for BBI, extrapolated from a fault model used for EMFI.
%The objective of this chapter is to provide an explanation of the mechanisms and causes of faults in integrated circuits that are subjected to body biasing injection.
%Electrical models presented in the chapter \ref{chap:3icModeling} can be used to explain how electrical charge displacement in the IC during a BBI pulse allows changing some logic gates output values.
%Targeting an IC via BBI forces electric charges to be injected (resp. absorbed) with positive pulses (resp. negative).
%Therefore, it is possible to target a critical time in the IC calculation thanks to the ability to finely control the induced disturbances.
%Eventually, to verify the correctness of the proposed analysis, both substrate charge propagation and logic gate behavior studies will be conducted in parallel.

%This chapter introduces a fault model for BBI, explaining how and why faults occur in ICs subject to body biasing injection.
%With chapter \ref{chap:3icModeling}'s electrical models as a basis, it allows explaining how electrical charges displacement in the IC during a BBI pulse allows changing logic gates output values.
%Therefore, if an attacker has the ability to finely control these disturbances, they can target critical time.
%In order to verify the correctness of this analysis, both substrate charges propagation and logic gates behavior study are conducted.