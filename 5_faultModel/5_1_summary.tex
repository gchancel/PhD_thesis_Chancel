
\section{Summary \ddc}
\label{chap5:sect:summary}
%In this chapter, we introduce a fault model for BBI.
%It aims at explaining how and why faults occur in ICs subject to body biasing injection.
%Thanks to chapter \ref{chap:3icModeling}'s electrical models as a basis, it allows explaining how electrical charges displacement in the IC during a BBI pulse allows changing some logic gates output values.
%Therefore, thanks to the ability to finely control the induced disturbances, it is possible to target critical time in the IC calculation.
%Eventually, to verify the correctness of the proposed analysis, both substrate charges propagation and logic gates behavior studies are conducted.
In this chapter, we present a fault model for BBI.
The objective of this chapter is to provide an explanation of the mechanisms and causes of faults in integrated circuits that are subjected to body biasing injection.
The chapter\ref{chap:3icModeling} electrical models can be used to explain how electrical charge displacement in the IC during a BBI pulse allows changing some logic gate output values.
Therefore, it is possible to target a critical time in the IC calculation thanks to the ability to finely control the induced disturbances.
Eventually, to verify the correctness of the proposed analysis, both substrate charge propagation and logic gate behavior studies will be conducted.

%This chapter introduces a fault model for BBI, explaining how and why faults occur in ICs subject to body biasing injection.
%With chapter \ref{chap:3icModeling}'s electrical models as a basis, it allows explaining how electrical charges displacement in the IC during a BBI pulse allows changing logic gates output values.
%Therefore, if an attacker has the ability to finely control these disturbances, they can target critical time.
%In order to verify the correctness of this analysis, both substrate charges propagation and logic gates behavior study are conducted.