
\newpage
\section{Conclusion \ddc}
In this chapter, I introduced the interest of thinning the substrate of integrated circuits in a body biasing injection context.
Various works have studied this topic concerning laser fault injection \cite{lfiThinning}, and because the substrate is the physical environment where the energy is carried to the IC when performing \bbi, I decided to take a deep look on substrate thinning in this context.

In the first place, I ideally analyzed theoretical \bbi effects thanks to a geometric model, which allowed me to conclude that the resolution of \bbi not only depends on the substrate thickness but on the couple thickness/voltage.
On the one hand, thinning the substrate and reducing the voltage pulse amplitude should lead to identical results on a fixed target.
On the other hand, thinning the substrate while keeping the same voltage leads to a spread out \bbi susceptibility area.
Afterward, I conducted the same protocol using the electrical models presented in the chapter \ref{chap:3icModeling}.
Using an analogous analysis as in that chapter, I observed that the thinner the IC, for a fixed set of parameters, the higher the voltage is focused under the probe.
Therefore, the susceptibility area increases, which means that less voltage, therefore a less powerful generator, is required to create a very similar effect.
These outcomes were verified thanks to actual experiments on three identical ICs with their substrate thinned to various amounts.
Eventually, the main thing to remember concerning substrate thinning and \bbi is that, as opposed to \lfi, thinning the substrate is not necessary while the attacker has a powerful enough generator.
However, sometimes, thinning the substrate can financially be more interesting than purchasing a more powerful pulse generator.

%In this chapter, I introduced the interest of thinning the substrate of integrated circuits when performing Body Biasing Injection.
%In the first place, I presented a geometric approach which allowed me to take a step back from electric simulations.
%This approach brought mathematical relations allowing to evaluate preliminary the effects of thinning the substrate of a target IC.
%Thanks to this approach, I drew three conclusions on the implications of substrate thinning on \bbi.
%Then, thanks to the previous simulation flow introduced in Chapter \ref{chap:3icModeling}, I performed simulations allowing me to complete the geometric approach.
%This allowed me to verify the soundness of the geometric conclusions.
%After that, I introduced substrate thinning in practice, including a quick look of the tools commonly used.
%Eventually, I presented practical experiments on actual thinned ICs, and I was able to verify the soundness of both the geometric and electric approaches.
