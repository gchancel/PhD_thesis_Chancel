
\section{Summary \ddc}
In this chapter, I study the interests of thinning the substrate of integrated circuits in a \bbi context.
Although this topic had been studied for \lfi in the past \cite{lfiThinning}, it was not the case for \bbi at the beginning of my thesis.

To that end, I divided the work into three parts.
First, I present a geometric approach to IC substrate thinning, allowing a high degree of abstraction from electronics.
Second, thanks to the models I introduced in the Chapter \ref{chap:3icModeling}, I introduce simulation results for various substrate thicknesses, allowing to verify the soundness of the geometric approach.
Eventually, I present actual experiments performed on three identical IC targets where their substrate have been thinned to different amount.

%This chapter proposes to study the interests of thinning the substrate of integrated circuits with the aim to enhance Body Biasing Injection efficiency.
%First, we are going to present a geometrical approach in order to appreciate with a certain abstraction from electronics the effects of substrate thinning on ICs behaviors.
%Second, thanks to the models presented in Chapter \ref{chap:3icModeling}, in addition to the geometrical approach, we are going to theoretically analyze the effects of substrate thinning from an electrical point of view.
%Eventually, in order to verify the soundness of the geometric approach and the simulation results, experiments are going to be studied thanks to an actual analysis of substrate thinning on identical IC targets behavior.
